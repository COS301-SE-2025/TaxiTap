\documentclass[a4paper,12pt]{article}
\usepackage[utf8]{inputenc}
\usepackage{longtable}
\usepackage{geometry}
\usepackage{hyperref}
\usepackage{array}
\usepackage{caption}
\usepackage{booktabs}
\geometry{margin=1in}
\usepackage{graphicx}
\usepackage{float}

\title{Coding Standards Document\\Taxi Tap by Git It Done}
\date{}

\begin{document}

\maketitle

\begin{figure}[H]
  \centering
  \includegraphics[width=0.5\textwidth]{LogoGroup.png} 
\end{figure}

\begin{figure}[H]
  \centering
  \includegraphics[width=0.5\textwidth]{LogoTaxiTap.png} 
\end{figure}

\newpage

\tableofcontents
\newpage

\section{Introduction}
This document outlines the coding standards, conventions, and practices adopted for the development of the project. Adhering to these standards ensures:

\begin{itemize}
\item \textbf{Uniformity}: All developers follow the same style.
\item \textbf{Clarity}: Code is easy to read and maintain.
\item \textbf{Reliability}: Reduce bugs caused by inconsistent practices.
\item \textbf{Efficiency}: Improve development speed through predictable patterns.
\end{itemize}

\section{Repository Structure}
The project repository follows a clear and organized structure:

\begin{verbatim}
project-root/
|-- .github                     % Workflow files
    |-- workflows
        |-- platform.yml
|-- assets                      % All project images
    |-- images
|-- docs/                       % Documentation
|-- platform                    % The app
    |-- app                     % Frontend
    |-- assets                  % Fonts and images
    |-- components
    |-- constants
    |-- contexts
    |-- convex                  % Backend
    |-- hooks
    |-- tests/                  % Unit and integration tests
    |-- .eslint.config.mjs      % ESLint configuration
    |-- package.json            % Project dependencies and scripts
|-- .prettierrc                 % Prettier configuration
|-- README.md                   % Project description
\end{verbatim}

\section{Coding Conventions}

\subsection{Language and Framework}
The project uses:
\begin{itemize}
\item Programming Language: \texttt{TypeScript/JavaScript}
\item Frontend Framework: \texttt{React Native}
\item Backend Framework: \texttt{Convex}
\end{itemize}

\subsection{Naming Conventions}
\begin{itemize}
\item \textbf{Files and directories}: camelCase and PascalCase
\item \textbf{Variables}: camelCase
\item \textbf{Constants:} \texttt{UPPER\_SNAKE\_CASE} and camelCase
\item \textbf{Classes/Components}: camelCase and PascalCase
\item \textbf{Functions}: camelCase
\end{itemize}

\subsection{Code Style}
\begin{itemize}
\item Indentation: 2 spaces
\item Maximum line length: 100 characters
\item Use semicolons at the end of statements
\item Use single quotes for strings
\item Always use curly braces for blocks
\end{itemize}

\section{Tooling and Configuration}

\subsection{Prettier}
Prettier is used for consistent code formatting.
\begin{verbatim}
npm install --save-dev prettier
\end{verbatim}

Example \texttt{.prettierrc} configuration:
\begin{verbatim}
{
  "semi": true,
  "singleQuote": true,
  "tabWidth": 2,
  "printWidth": 100,
  "trailingComma": "all"
}
\end{verbatim}

\subsection{ESLint}
ESLint is used for linting and enforcing coding standards.
\begin{verbatim}
npm install --save-dev eslint
\end{verbatim}

\section{Testing Standards}
\begin{itemize}
\item All code must be covered with unit tests using \texttt{Jest}.
\item Use descriptive test names.
\item Mock external dependencies where appropriate.
\item Integration tests.
\end{itemize}

\section{Documentation Standards}
\begin{itemize}
\item Use clear and concise comments where necessary.
\item Keep the README updated with setup and usage instructions.
\end{itemize}

\section{Pull Request Guidelines}
\begin{itemize}
\item Ensure code passes all linting and tests before submission.
\item Provide clear commit messages.
\item Perform peer reviews.
\item Do not merge code with failing tests.
\end{itemize}

\section{Continuous Integration}
\begin{itemize}
\item We are using GitHub Actions to automate linting and tests on every pull request.
\end{itemize}

\end{document}